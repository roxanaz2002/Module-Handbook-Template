\documentclass{report}
% 当然也可以用article,也可以写自定义的cls文件
\usepackage{course_template} 
\usepackage{graphicx} %figure support

\begin{document}
\begin{titlepage}
    \centering
    \includegraphics[width=0.7\textwidth]{logo.png}\\[3cm] % width percentage, logo
    {\LARGE\textbf{Junior Wizard}} \\[1cm] 
    {\LARGE\textbf{Animagus Program}} \\[1cm]   
    {\LARGE\textbf{Module Handbook}} \\[1.5cm]
    \vfill % Pushes the content below to the bottom of the page
\end{titlepage}

\newpage 
\section*{Credit Conversion Rules}
\noindent
At \textbf{Hogwarts School of Witchcraft and Wizardry}, \textbf{1 credit} represents \textbf{16 Academic Hours} for standard lectures. However, for courses such as \textbf{Potions}, \textbf{Transfiguration}, \textbf{Defense Against the Dark Arts}, or \textbf{Quidditch Training}, \textbf{1 credit} equals \textbf{48 Academic Hours}. 
After all, brewing a perfect potion or dodging Bludgers takes more effort than waving a wand! So remember: more hours, more magic!

\medskip % 设置段间距
\noindent
The Animagus Program’s Degree required on Graduation is \textbf{141 credits}. Therefore, \textbf{1 credit} at \textbf{Hogwarts School of Witchcraft and Wizardry} is equivalent to approximately \textbf{1.5 ECTS} credits.

\medskip
\noindent
Each academic year consists of two semesters: Fall and Spring. During summer vacation, junior students are strictly prohibited from using magic on Muggles, with violations leading to serious consequences from the \texxtbf{Ministry of Magic}.
% \medskip
% \noindent
% The abbreviation \textbf{“IP”} in the “Grade” column stands for \textbf{“in progress.”}
%
\medskip
\noindent
\textbf{Academic Integrity:} 
\\
% 可以选择学术诚信条款
All courses place a high emphasis on academic integrity and strictly prohibit plagiarism, cheating, and other academically dishonest behaviors. Each course follows the university's plagiarism policies.

%%%%%%%%%%%%%%%%%%%%%%%%%%%%%%%%%%%%%%%%%%%%%%%%%%%%%%%%%%% outline %%%%%%%%%%%%%%%%%%%%%%%%%%%%%%%%%%%%%%%%%%%%%%%%%%%%%%%%%%%%%%%%%%
\newpage
\section*{Module Contents}

% 使用 longtable 创建支持分页的表格
\begin{longtable}{|p{3cm}|p{7.5cm}|p{1.5cm}|p{1.5cm}|}
\hline
\textbf{Course Code} & \textbf{Course Title} & \textbf{Credit} & \textbf{ECTS} \\ \hline
\endfirsthead % 定义表格的首页

% 续页表头(如果表格跨页)
\hline
\textbf{Course Code} & \textbf{Course Title} & \textbf{Credit} & \textbf{ECTS} \\ \hline
\endhead
%%% 不同课程模块,以下为示例
% 课程代码 & 课程名称 & 学分 & ECT 
% Mathematics 
\multicolumn{4}{l}{\textbf{Mathematics}} \\ \hline
MATH112 & Linear Algebra I & 4 & 6 \\ \hline
MATH120 & Discrete Mathematics & 4 & 6 \\ \hline
\textbf{Total} & & & \textbf{12}\\ \hline

\multicolumn{4}{|l|}{\textbf{Core Magic Subjects}} \\ \hline
MAG101 & Charms & 4 & 6 \\ \hline
MAG102 & Transfiguration & 4 & 6 \\ \hline
MAG103 & Defense Against the Dark Arts & 4 & 6 \\ \hline
\textbf{Total} & & & \textbf{18} \\ \hline

\multicolumn{4}{|l|}{\textbf{Alchemy and Potions}} \\ \hline
POT201 & Potions & 4 & 6 \\ \hline
ALC202 & Advanced Alchemy & 4 & 6 \\ \hline
\textbf{Total} & & & \textbf{12} \\ \hline

\multicolumn{4}{|l|}{\textbf{Advanced Magic and Forbidden Arts}} \\ \hline
FOR401 & Ancient Runes & 3 & 4.5 \\ \hline
FOR402 & Dark Magic Studies & 4 & 6 \\ \hline
\textbf{Total} & & & \textbf{11} \\ \hline

% Electrical Engineering 
\multicolumn{4}{l}{\textbf{Electrical Engineering}} \\ \hline
EE131   & Electromagnetics & 4 & 6 \\ \hline
EE156   & Signals and Systems & 4 & 6 \\ \hline
EE180   & Introduction to Control & 4 & 6 \\ \hline
\textbf{Total} & & & \textbf{18}\\ \hline

% Computer Science 
\multicolumn{4}{l}{\textbf{Computer Science}} \\ \hline
CS150   & Introduction to Programming & 4 & 6 \\ \hline
CS113   & Introduction to Machine Learning & 4 & 6 \\ \hline
\textbf{Total} & & & \textbf{12}\\ \hline

\end{longtable}



\end{document}
